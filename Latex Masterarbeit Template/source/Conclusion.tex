\chapter{Conclusion}
\label{chp:conclusion}

The last chapter of this thesis is intended to summarize the whole topic and discuss whether the completion of the initial goals was achieved. It also proposes ideas for further improvement and possible elaborations of the topic.

\section{Goals And Results Evaluation}
First of all, the goals of the game and after that the goals of the thesis will be summarized. The aim of this thesis was to solve the problem of how power is distributed in a parallel hybrid car having an electric motor and a traditional internal combustion engine. This optimization problem can be solved with game theory, where the engine and the electric motor are two players in a cooperative game, whose aim is to distribute the power in the most efficient manner. The primary goal of the game itself was defined as meeting the driver demands of the FTP75 drive cycle, while the secondary goals of the game were to minimize fuel consumption and exhaust emissions of CO, HC and NOX while maintaining a target SOC of the battery. All of these goals were addressed in the payoff function, which involves the torque deviation, the fuel consumption rate, the gas emissions and the battery SOC target deviation. Since the game is cooperative, the two players can cooperate with each other in order to achieve a better outcome. Different game-theoretical cooperative concepts were applied and their results were compared with each other. These were Pareto Optimality, Nash Equilibrium as in non-cooperative game theory, Nash Bargaining solution, Kalai-Smorodinsky solution, the core and the Shapley value. 

In order to develop such a game and to solve it for every time step of the FTP75 drive cycle, firstly a model of a hybrid car was required for the simulation. This model included Controllers for the various hybrid vehicle components which were all explained - Drive Cycle Controller, Power Controller, Engine Controller, Electric Motor Controller, Transmission Controller and Vehicle dynamics Controller. The game-theoretical solution of the power distribution was calculated in the Power Controller using an embedded Matlab function. All parameters of the hybrid vehicle for the model in Simulink are taken as in the Toyota Prius second generation. The game was dynamically simulated and the solution was computed for every time step (second) during FTP75, whose total duration is 1874s. The simulation of the FTP75 drive cycle was split into 5 phases and the battery was recharged. During acceleration power was taken from the battery and also supplied by the engine. During deceleration, the regenerative braking power was stored in the battery; hence, charging it. The driver demands were met completely during the whole FTP75-2 and FTP75-4. The cases where it could not meet them was during FTP75-1, FTP75-3, FTP75-5 only where the SOC of the battery falls below the allowed 40 \%. At that time the battery was charged purposefully and the vehicle could not meet the demanded speed any more, which happened for 310s in total of the full 1874s. As soon as the battery was charged to the target 60 \%, the driver demands were adhered to again.

After the simulation all results can be generally concluded as follows. Regarding the first goal of meeting the driver demands, all game-theoretical approaches performed equally well. Moreover, the torque distribution problem between the engine and motor was solved most efficiently by the Nash Equilibrium, which produced the smallest changes of required torque for a time period, smoothing the solution. In addition, the fuel consumption and gas emission minimization were best handled by the Pareto Optimality, followed by the Nash solution and the Nash Equilibrium. With regard to the battery SOC goal, the Pareto Optimality and Nash solution maintained the highest SOC at the end of each of the FTP75 phases.

\section{Relation To State Of The Art Research}
In conclusion, the strengths of this thesis project are as follows. Firstly, a complete model of a hybrid car was built in Simulink. Secondly, the power distribution problem was solved with six different game-theoretical approaches. Thirdly, the game was simulated for the whole duration of 1874s (17mins) of the FTP75 drive cycle. These achievements are a crucial extension to the research described in \ref{sec:stateart}, where the major disadvantages were that all of the games are non-cooperative, very often the exact game-theoretical approach is sparsely described, including the most significant part - the payoff function, and that often no or little simulation results were shown.

\section{Future Work}
In the end, possible improvements are discussed. One possibility might be to extend the game from a two-player to three-player cooperative game, incorporating the battery as the third player. Thus, the core and the Shapley value, which are usually applied games with more than two players, will produce more meaningful results and the core will not be equivalent with the Pareto efficient outcomes. A further extension might be to simulate another drive cycle like the New European Driving Cycle (NEDC) and compare its results with FTP75. From the technical point of view, a the automatic gearbox  assumed in this thesis can be extended to manual gearbox with 5 gears, taking into account gear ratios. Furthermore, an ultra capacitor (UC) can be added to the conventional battery in order to boost its performance. Nevertheless, all of these extensions were assumed to be complicated and are therefore out of the scope of this thesis.
