\chapter{Introduction}
\label{chp:intro}
This chapter outlines the motivation for the choice of the topic and sets the scope and the goals of this thesis. Moreover, a brief history of the initial interest in hybrid vehicles and the state of the art is presented, concentrating on the latest literature.

\section{Motivation}
A crucial aspect while understanding the interest of an author in a particular topic is the motivation. Due to the previous completion of two seminars in the area of Game Theory, the author developed a genuine interest in this topic. Another reason comes from the fact that a lot of real-life situations can be represented as games whose aim is to decide whether to invest in a company in a particular economic situation. They can be solved by optimal approaches which are mathematically defined and usually present function maximization or minimization problems. The first seminar, which the author completed, dealt with Pursuit-Evasion games with UAVs (Unmanned Aerial Vehicles) and was an essential introduction to the fundamentals of Game Theory for the author, who had absolutely no experience in the area beforehand. The second seminar was specifically pointed at this thesis and coped with the problem of power management in hybrid vehicles from a game-theoretical point of view. Thus, a portion of the necessary mathematical and theoretical knowledge for completing the goals of this thesis was acquired prior to commencing the project.

\section{Goals}
Before starting a thesis, the various goals need to be set. First of all, even before defining what game is to be played, a hybrid vehicle model has to be developed. It must include all essential components of a hybrid car. For this reason a lot of parameters need to be taken from a real hybrid vehicle, for example, the Toyota Prius hybrid car. This car was chosen, since it was the first mass-produced hybrid vehicle in the world, whose manufacturing started in 1997. The hybrid model of a car can be implemented in Simulink where it will be simulated later. Secondly, the game to be played must be formally defined. It needs to be decided how many players are going to take part in it, what kind of a game it is - cooperative or non-cooperative, according to whether the players are allowed to make agreements between themselves during the game. Thirdly, after the game definition, plausible solution approaches have to be chosen which can be applied to solve the game at every stage (time step). Thorough examination of at least 2 or 3 game-theoretical solution approaches is required so that their results can be compared. In the end, not only 2 or 3, but 6 game-theoretical approaches were employed. Afterwards, the solutions have to be implemented as algorithms and each solution approach has to produce a single solution at every stage. Lastly, the defined game is dynamically simulated and solved at each time step. The first and most crucial goal of the game is to achieve the driver demanded speed, while minimizing fuel consumption, gas emissions and maintaining a target state of charge (SOC) of the battery, which all together form the second goal of the game. Regarding the simulation, a whole established drive cycle is desired instead of a user-defined speed versus time demand. To summarize, the four primary goals of the thesis are:

\begin{itemize}
\item Create hybrid vehicle model
\item Specify game-theoretical definition
\item Develop game-theoretical solution approaches
\item Simulate game
\end{itemize}


\section{State of the art} 
\label{sec:stateart}
This section gives a brief introduction to the history and the literature on hybrid vehicle in the past 10 years.

The first hybrid vehicle was built in 1900 by Ferdinand Porsche. The car was called Lohner-Porsche Mixte Hybrid. It was a serial hybrid car with two wheel motors, it had a 5.5 liter 18kW engine, a battery and an electric generator. The first mass produced hybrid car was the Toyota Prius from 1997 manufactured in Japan.

The first research papers about hybrid vehicles date back from the 1970s. \citet{lafrance1973electrical} give an overview of one serial configuration and two parallel configurations - with a single and with a dual motor. They compare the suitability of different electric motor designs for various driving conditions.

Although there has been a lot of research in hybrid vehicles in the last two decades, there is a limited number of approaches which consider the power management problem from a game-theoretical point of view.

%-----------------------------------------------

First of all, \citet{gielniak2004power} describe a fuel cell hybrid electric vehicle and solve the power distribution problem as a two-player non-cooperative game. The vehicle has a fuel cell, battery, ultra capacitor and two 35 kW motors. The fuel cell tries to maintain a target 60\% State of Charge (SOC) of the Energy Storage Subsystem (ESS) or the battery. Power is firstly taken from the fuel cell and then the rest is taken from the ESS. The voltage of the ESS depends on the SOC and it is being cooled down. The Ultra Capacitor (UC) is a lumped energy storage device with air cooling and its SOC depends on the UC voltage. The motor has a maximum bus power limit. In addition, there is a transmission model and accessory loads. The accessories such as air conditioner and power steering are powered from the electric motor.

The game-theoretical approach involves two players - all power supplying components as the first player and all power consuming components as the second player. The Master Power Management Controller (MPMC) governs all components in the powertrain and calculates the solution. The objective of the game is to save fuel and to accelerate fast. The decisions of the players are how much power to supply at a given moment. The payoff of each player is represented by utility functions - efficiency, performance and composite utilities. The efficiency utility is a function of the strategy of the opponent - how much power they contribute. The performance utility in this case is acceleration. The composite utility is then computed for a time moment after all components have been mapped to an efficiency or performance function. The authors \citet{gielniak2004power}, however, do not provide any insights on how exactly the game-theoretical solution was computed.

Regarding simulation three drive-cycles were examined - Federal Test Procedure (FTP), US06 and Constant 60 Miles Per Hour. A simulation tool called ADVISOR \citep{burch1999advisor} is used. A comparison is drawn between a Basic control strategy without game theory and a game-theoretical control strategy. In the US06 cycle, the Game Theory control achieves more miles per gallon than the Basic control. Regarding acceleration the Basic control is able to accelerate from 0 to 85 miles in 17.48s, whereas the Game Theory control needs only 11.77s.

%-----------------------------------------------

A further approach towards power control is presented by \citet{chin2010design}. The hybrid is based on the Toyota Prius and has a gasoline engine and an electric motor which are the two players in a bimatrix game. The vehicle configuration consists of a Gasoline Engine Controller, Electric Motor Controller and a Power Controller as the main unit for the computation of the solution. On the one hand, the Gasoline Engine Controller manages the fuel and air injection. A three-way catalyst system is utilized for the gas emissions of $HC$, $CO_2$, $CO$ and $NO_x$. On the other hand, the Electric Motor Controller also includes battery and capacitor units. Batteries have high energy capacity but cannot provide much power, whereas capacitors have less energy capacity, but can produce a lot of power. Capacitors help improving battery lifetime and are also useful for providing short bursts of power during acceleration. The Power Controller is responsible for computing the game theory solution. It takes the requested torque from the driver and determines the optimal strategies for the engine and the electric motor. It outputs the torque for each of the two power sources. In addition, it charges the battery when the SOC is low. The goal of the Power Controller is to minimize fuel usage and at the same time maximize torque and reduce gas emissions. This hybrid configuration also has different modes of operation - Engine only, Motor only, Mixed and Battery Charge.

The game-theoretical solution consists of a non-cooperative bimatrix game. The payoff matrices for both players - the engine and the electric motor, are of size $M \times N$. These denote the strategies of player 1 and 2. The game is solved by a Nash Equilibrium \citep{nash1951non}. When player 1 chooses strategy $i \in M$ and player 2 chooses strategy $j \in N$ the payoff is $(a_{ij},b_{ij})$ and it constitutes the Nash Equilibrium if this pair contains the optimal strategies for both players. Pure strategies are extended to mixed by specifying the strategies as a vector of probabilities over all pure strategies. The strategies are how much torque to contribute, measured according to the revolution speed from 0 to 6000rpm. Each payoff entry in the matrix is a function of fuel consumption, gas emissions, engine temperature, SOC deviation, extra weight and driver's demands. The Nash Equilibrium of the game is computed by the Lemke-Howson algorithm \citep{lemke1964equilibrium} in the Power Controller. However, a major weakness is that no simulation results are presented.

%-----------------------------------------------

\citet{dextreit2014game} use the hybrid configuration of the Jaguar Land Rover Freelancer2. It has a diesel engine, two electric motors and a six-speed dual clutch transmission. The first electric motor is attached to the engine Crankshaft Integrated Starter Generator (CISG), while the second is attached to the rear wheels and is called the Electric Rear Axle Drive (ERAD). There are five driving modes. The EV mode is when only the ERAD provides torque, the Engine-only mode is when only the engine supplies torque. The parallel mode means that both the engine and the motors provide torque. There is a charging mode in which the engine produces the driving torque and the CISC torque. The last mode is a boosting mode, where the engine supplies the additionally needed driving torque.

The game-theoretical approach \citet{dextreit2014game} adopt is a finite-horizon non-cooperative game, where the two players are the driver and the powertrain. The cost function penalizes fuel consumption, $NO_x$ emissions and the battery SOC deviation. In a typical dynamic programming approach the cost is a function of a state vector, a control vector and a vector of operating conditions. The driver chooses the operating conditions - requested wheel speed and wheel torque. The powertrain chooses the control variables and the state vector is the SOC of the battery. The game is solved using a feedback Stackelberg equilibrium \citep{von1952theory}. Firstly, the game is solved statically where the first player is the leader, who maximizes a function $J(w,t)$ by selecting the powertrain operating demands $w(t) \in W$ . The second player (the follower) is able to observe the first player's decision and depending on that it selects its control vector $u(t) \in U$. It is assumed that both players make rational decisions. The pair $(w^*, u^*)$ is the Stackelberg equilibrium.
%\[
%J(u^*,w^*) = max_{w \in W} min_{u \in U} J(u,w)
%\]
Next, the dynamic game is described which consists of $(T-1)$ stages, also called the horizon of the game. Given the initial state vector $x(0)$ the follower chooses their move $w(0)$. Then the follower selects their move $u(0)$. Thus, after the first stage the state is $x(1) = f(x(0),u(0),w(0))$. This continues until the last $(T-1)$ stage of the game.
%\[
%J(u^*(\cdot),w^*(\cdot) = max_{w(0)} min_{u(0)} max_{w(1)} min_{u(1)} ... max_{w(T-1)} min_{u(T-1)} %J(u(\cdot),w(\cdot))
%\]
%The Game Theory control has also the following cost function
%\[
%L(x,u,w) = \alpha \cdot Fuel + \beta \cot NO_x + \mu \cdot [ SOC_{setpoint} - SOC]^2 + \gamma \cdot G(w)
%\]
The GT controller \citet{dextreit2014game} implemented computes the state, control and operating values offline in three modules - GT Maps, Mode Arbitration and Mapping to torque demand. In the GT Maps module the wheel torque and speed, gear and battery SOC are discretized. The wheel torque is in the range from 0 to 1000 Nm. The wheel speed is between 0 and 100 rad/s. The battery SOC is between 40\% and 70\%. As output the module produces two modes to choose from. The Mode Arbitrator chooses one of these, for example the EV or the parallel mode. The aim of the Mapping to torque demand module is to distribute the torque between the engine and the motors.

\citet{dextreit2014game} present a baseline controller solved using Dynamic Programming and compare it with the game theory controller. Measurements are conducted over three drive-cycles for $CO_2$ emissions, equivalent to fuel consumption, $NO_x$ emissions and deviation of SOC. These show that the GT controller outperforms the baseline controller.

%-----------------------------------------------

There is another similar approach to the one of \citet{dextreit2014game}. The idea is extended by \citet{chen2014game} who describe a single-leader multiple-follower game, where the follower is not only one (the battery) like in \citet{dextreit2014game}, but also include other auxiliaries. The vehicle is a heavy-duty truck with an electric refrigerated semi-trailer and a battery. The aim is to manage the battery power and the refrigerated semi-trailer power. The vehicle model consists of the internal combustion engine. The total power which is requested from it is a sum from the driving power and the power for all auxiliaries. The semi-trailer model considers the air transfer from inside, outside and the cooling power. The problem is formulated as minimization of fuel consumption over a time interval. Energy balance is sought so that the battery energy and the air temperature in the semi-trailer at the end of the drive cycle are the same as in the beginning.

The game-theoretical approach assumes a single leader - the driver and multiple followers - all auxiliaries. The leader selects their action $w(t) \in W$, which is the demanded wheel torque and speed. Then each follower $f_i$ chooses their action $u_i(t)$. The game is split in two levels. The first level considers $N$ two-player games between the leader and each of the $N$ followers. It solves the games offline with a Stackelberg equilibrium where the driver maximizes and the leader minimizes the cost function. The strategies are stored in a lookup table for each follower. The second stage is computed online and it combines all followers who play their Stakelberg strategies as stored in the lookup table. They all have to reach a mutual decision and this can be solved by a Nash Equilibrium. The central computation happens in the Energy Management System Operator (EMSO). It gathers the actions from all followers and sends them back the computed overall strategy for this stage. This happens until an equilibrium is reached. For simulation \citet{chen2014game} compare two control strategies - optimizing only the battery and optimizing both the battery and the semi-trailer power with the game theory approach. The optimized battery and trailer achieve better fuel consumption by 0.08\%.

%-----------------------------------------------

After exploring the literature, it can be concluded that all game-theoretical approaches assume a non-cooperative game and solve it either by a Nash Equilibrium or by a Stackelberg Equilibrium. Mostly, two players are involved in the game methods proposed. In two of the approaches above the two players are the driver and the powertrain. Two other cases described the two players, being divided into the power-consuming and power-supplying devices in the powertrain or into the electric motor and the engine. Nevertheless, the most crucial point to bear in mind is that that no solution approach applies a cooperative game. This thesis defines the game to be cooperative and this makes its topic totally unexplored and innovative.

\section{Content}
The thesis is divided into the following chapters. Chapter one, the Introduction, already described the topic from the point of view of the motivation of the author, the goals which are set for completing the project successfully and the state of the art literature. Chapter two, Fundamentals, moves the focus to the theoretical knowledge required for the thesis. It introduces the concept of a typical hybrid vehicle and presents the configuration of one hybrid vehicle, the Toyota Prius. Then, the game definition is given and six different game-theoretical solution approaches, applicable to cooperative games, are described - Pareto Optimality, Nash Equilibrium, Nash Bargaining solution, Kalai-Smorodinsky solution, Core and Shapley value. Chapter 3, Implementation, outlines the hybrid model in Simulink and each of the different Controllers which are responsible for the components of the vehicle. Then, it goes into details regarding the game strategies and payoffs computation and the algorithms used to implement each of the six game-theoretical approaches in Matlab. Chapter 4, Simulation, describes the specific configuration parameters during the simulation of the model in Simulink. Chapter 5, Results, discusses the game theory solutions graphically by comparing them. Moreover, it shows the Simulink scopes in order to evaluate the accuracy of the results, particularly focusing on the goals of the game - whether the drive cycle demanded speed is achieved, fuel and gas emissions are minimum and battery target SOC is maintained. Chapter 6, the Conclusion, summarizes the thesis by discussing its strengths and weaknesses and proposing further ideas for improvement.