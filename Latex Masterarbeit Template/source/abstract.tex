\begin{otherlanguage}{english}
\begin{abstract}

This thesis proposes a new approach for solving the power distribution problem in a hybrid vehicle. The main reason for driving hybrid cars is that they are environmentally friendly, because they save up to 50\% fuel in comparison with conventional cars with the same parameters. In addition, gas emissions can further be reduced by up to 50\% in hybrids. A typical parallel hybrid faces one major problem of power distribution or how to most efficiently split the torque between the internal combustion engine and the electric motor, which can work together in parallel. For this purpose game-theoretic solution approaches are examined and a cooperative game between two players - the engine and the motor, is defined. Both players are allowed to make agreements between themselves by cooperating with each other. The strategies of the players in the game represent how much torque each player contributes at each time step. The utility functions of the players are minimized and are therefore regarded as penalties. They take into account torque deviation from the required torque, fuel consumption, CO, HC and NOX emissions and the State of Charge (SOC) of the battery. These represent the goals of the game - meeting the demanded vehicle speed, while minimizing both fuel consumption, gas emissions and maintaining a target SOC of the battery. Six game-theoretical approaches applicable to cooperative game theory are explained, implemented and simulated - Pareto Optimality, Nash Equilibrium, Nash Bargaining solution, Kalai-Smorodinsky solution, Core and Shapley value. A Simulink model of a hybrid vehicle is simulated during the Federal Test Procedure 75 (FTP75) drive cycle with all six game-theoretic approaches by solving the game at every stage. The results show that the torque distribution problem was best solved by the Nash Equilibrium, whereas Pareto Optimality and Nash solution performed most optimally at minimizing fuel consumption and exhaust emissions. To conclude, this thesis examines the topic of hybrid vehicles, which has been around for two decades, but applies a totally new approach of cooperative game theory and explores six different ways of solving the game, simulating a complete drive cycle with total duration of 31.24 minutes.
  
\end{abstract}
\end{otherlanguage}
