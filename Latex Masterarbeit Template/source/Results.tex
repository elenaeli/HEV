\chapter{Results}
\label{chp:results}

This chapter discusses the simulation results of the five drive cycles phases. For each of them different scopes are presented, from which the most important are the Game Theory Scope, which shows the distribution of motor and engine torque. Then, the Drive Cycle and the Power Controller Scopes are the next in order of significance because they show how the driver demands are met and the SOC of the battery along with the speeds of the engine, motor and generator. The other important scope are for the Engine and Motor since they show speed, torque and power. For the first two solution approaches, all of the scopes will be shown for the 5 phases of FTP75 and for the next four solution approaches only the Game Theory scope will be displayed because it is the most significant.

\section{Pareto Optimality}
This section deals with the first game-theoretical solution and the corresponding results. 

\section{Nash Equilibrium}
This section shows the Game Theory Scope, Drive Cycle Scope, the Fuel Emissions Scope and the Power Controller Scope of all of the five phases in the FTP75 drive cycle.

FTP75-1 results are displayed in the Figures \ref{fig:gtne1}, \ref{fig:dcne1}, \ref{fig:pcne1}, \ref{fig:ene1} and \ref{fig:fene1} in the appendix. The pink line in \ref{fig:dcne4} is the demanded speed and the yellow is the actual speed. As it can be seen from Figure \ref{fig:gtne1} the torque demand is distributed between the motor and engine. The engine can provide torque only in the range between 83-136 \textit{Nm}. In the other cases between -400 and 400 \textit{Nm} the motor can contribute without any limits. When the engine starts giving torque, the motor torque is decreased. At time 275-330 the torque demand rises to 400 \textit{Nm} because the battery SOC falls below 40 \% as seen in Figure \ref{fig:pcne1}. At the same time the difference between demanded and actual speed grows up to a maximum of 70 \textit{km/h}; hence, the acceleration reaches its maximum of 1. At time 230-260s the engine revolution speed jumps up and down in a range from 2800-3300 \textit{rpm} or 293-335 \textit{rad/s}. Therefore, a hysteresis was implemented in this range in order to keep the revolution speed either at 2800 or at 3300 instead of letting it jumping in between. Nevertheless, the results are still not completely satisfactory, since the engine torque and power as in Figure \ref{fig:ene1} goes up and down very quickly, although the revolution speed is restricted as much as possible. For the same reason the emissions and the fuel consumption are also unstable in the same time region as in Figure \ref{fig:fene1}.

The FTP75-2 results are in Figure \ref{fig:gtne2}, \ref{fig:dcne2}, \ref{fig:pcne2}, \ref{fig:ene2} and \ref{fig:fene2}. They show that in the second phase of the drive cycle there is a much frequent use of the engine compared to the first phase. Moreover, the motor requests negative torque much more often. This is due to the fact that in Figure \ref{fig:dcne2} there are more decelerations than in Figure \ref{fig:dcne1}. Since the battery does not have to be recharged in FTP75-2, the drive cycle demands are met very well and the actual and demanded speed are always close. Therefore, this phase also runs very fast compared to FTP75-1.

The FTP75-3 results are shown in \ref{fig:gtne3}, \ref{fig:dcne3}, \ref{fig:pcne3}, \ref{fig:ene3} and \ref{fig:fene3}. In this phase the SOC again falls below 40 \% and then the difference between actual and demanded speed becomes at most 25 \textit{km/h}. This happens at time 120s and thus the acceleration is increased to the maximum of 1 and the required torque becomes 400 \textit{Nm}. When the battery is recharged to 60 \% at 180s, the actual speed still undergoes a lot of changes until it meets the required speed at time 240s. The reason is that the acceleration has been constantly 1 for a very long and then it suddenly drops to -1. The motor, engine and generator revolution speeds are also kept constant during recharge mode. Furthermore, in Figure \ref{fig:ene3} it can be seen that the problem with the rapid change of torque from the engine does not occur.

The FTP75-4 results are in \ref{fig:gtne4}, \ref{fig:dcne4}, \ref{fig:pcne4}. It can be concluded that in this phase the drive cycle demands are again met very accurately since there are almost no differences in the demanded and actual speeds in Figure \ref{fig:dcne4}. The battery is both recharging during deceleration, corresponding to negative torque and discharging during acceleration with positive torque.

In the last phase, the FTP75-5, the battery has to be recharged in the 50s and the maximum difference of demanded and actual speed this incurs is 70 \textit{km/h}. At 120s the battery is already at the target 60 \% and the actual speed tries to keep up to the demand. However, it goes up and down before it comes to the right demand speed at 170s, similarly as in FTP75-3. 

\section{Nash Bargaining solution}
For the next four game-theoretical approaches only the Game Theory scope will be shown, since there is little difference in the other scopes. Moreover, it is crucial to examine the motor and engine torque distribution differences between the different game theory approaches.

