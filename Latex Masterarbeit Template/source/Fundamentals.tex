\chapter{Fundamentals}
\label{chp:basics}

\section{Lemke-Howson Algorithm}
One of the most popular algorithms for finding a Nash Equilibrium for bimatrix non-zero-sum games is the Lemke-Howson algorithm \citep{lemke1964equilibrium}. The Matlab implementation developed by \citep{lemkeHowson2014Matlab} was utilized to find one Nash Equilibrium in mixed strategies. This function contains a parameter, which affects the final result of the algorith. Changing the parameter yields different Nash equilibria as output. This parameter is the initial pivot, which is a number between 1 and $e$ + $m$, where $e$ and $m$ are the number of strategies of player 1 and player 2 respectively. This pivot is the label of the graph (strategy), which the Lemke-Howson algorithm drops first. There are two graphs for both players and they each have nodes and edges.

There a number of reasons for the different Nash Equilibrium solutions that the algorithm produces. Regarding to \cite{lemke1964equilibrium} there exist an odd number of Nash Equilibria in a nondegenerate game. A nondegenrate game is a game where no mixed strategy with a support of size $k$ has more than $k$ pure strategies \citet{nisan2007algorithmic}. Moreover, a support of a mixed strategy is defined as the set of pure strategies which have positive probabilities. Since there is an odd number of Nash Equilibria, there must be at least one Equilibrium, which proves that the algorithm will always find one solution in mixed strategies. However, which of all Nash Equilibria is found depends on which strategy label is dropped first. The initial pivot label which the algorithm drops first can belong to either of the two players and be any of their $e$ or $m$ strategies.

As discussed in \citep{shapley1974note} the Lemke-Howson algorithm possesses a significant weakness, namely that it is neither guaranteed that the algorithm will find all possible solutions, nor that it will tell if there are any unfound solutions.

The fundamentals of the Lemke-Howson algorithm are as follows. Let us assume a scenario of a 2-player bimatrix game as the one used throughout this thesis, where the players $E$ and $M$ each have $e$ and $m$ pure strategies and their payoffs are in the matrices $A$ and $B$ respectively. The mixed strategies are the vectors $s=(s_1,s_2...,s_e)$ and $t=(t_{e+1},t_{e+2},...,t_{e+m})$ where $S = {s \geq 0; \sum_{i=1}^{e}s_i = 1}$ and $T = {t \geq 0; \sum_{j=e+1}^{e+m}t_i = 1}$ are the sets for the mixed strategies spaces. Then, the payoff for player 1 is $\sum_{i=1}^{e}\sum_{j=e+1}^{e+m} a_{ij} s_i t_j$ and the payoff for player 2 is $\sum_{i=1}^{e}\sum_{j=e+1}^{e+m} b_{ij} s_i t_j$.
If we also define the sets:
\[
\tilde{S} =  S \cup \{ s \geq 0: \sum_{i=1}^{e} s_i \leq 1 \text{ and } \prod_{i=1}^{e} s_i = 0 \}
\]

\[
\tilde{T} = T \cup \{t \geq 0: \sum_{j=e}^{e+m} t_i \leq 1 \text{ and } \prod_{j=e}^{e+m} t_i = 0 \}
\]
this allows us to define closed convex polyhedral regions $S^k$ in $\tilde{S}$:
\[
S^i = \{ s \in \tilde{S}: s_i = 0 \} \text{ for } i \in \{1,...,e\}
\]

\[
S^j = \{ s \in S: \sum_{i=1}^{e} b_{ij} s_i = \max_{l \in \{e+1,...,e+m\}} \sum_{i=1}^{e} b_{il} s_i \} \text{ for } j \in \{e+1,e+2,...,m\}
\]

$S^i$ contains all $\tilde{S}-S$ and $S^j$ contains the mixed strategies for player 1 and the pure strategy $j$ of player 2 which gives him the best response. $S^i$ and $S^j$ both make $S^k$ and cover all of $\tilde{S}$. The same can be applied to define the regions $T^k$ in $\tilde{T}$:
\[ 
T^i = \{ t \in T: \sum_{j=e+1}^{e+m} a_{ij} t_j = \max_{l \in \{1,...,e\}} \sum_{j=e+1}^{e+m} a_{lj} t_j \} \text{ for } i \in \{1,...,e\} 
\]

\[
T^j = \{ t \in \tilde{T}: t_j = 0 \} \text{ for } j \in \{e+1,...,e+m\} 
\]

The Lemke-Howson algorithm represents the strategies of both players in two graphs with nodes and edges. The already mentioned definitions are required in order to define a labelling for the graphs. A labelling of a node consists of all of the labels of all surrounding regions of this node. Let the nonempty label of $s \in \tilde{S}$ be $L'(s) = \{ k: s \in S^k \}$ and similarly the nonempty label of $t \in \tilde{T}$ be $L'(t) = \{ k: t \in T^k \}$ and the label of the node pair with pure strategies $(s,t) \in \tilde{S} \times \tilde{T}$ be $L(s,t) = L'(s) \cup L''(t)$. A node pair $(s,t)$ is completely labelled whenever $L(s,t) = K$, meaning that it contains the labels for all regions $k \in K$. A node pair is almost completely labelled if $L(s,t) = K - \{k\}$ for some $k \in K$. 

Then, as in \citep{shapley1974note} a node pair $(s,t) \in S \times T$ is an equilibrium point of (A,B) if and only if $(s,t)$ is completely labelled. Let the two graphs be $G'$ in $\tilde{S}$ and $G''$ in $\tilde{T}$. Two nodes are adjacent if they are on the two ends of the same edge which means that their labels differ in exactly one element. The set of k almost completely labelled nodes in the graphs and their edges are the disjoint paths of the graph and cycles. The equilibria of the game are always located at the end of these paths. Also, the starting node, by default (0,0) is called an artificial equilibria and it is also located at the end of a path.

The algorithm works by starting at $(s,t) = (0,0) \in G' \times G'' $. Then a label $k$ to be dropped from $(s,t)$ is chosen (initial pivot parameter). This $k$ can belong to either $m$ or $e$. This means that a new node is reached by following an adjacent edge. Let this node be $(s,t)$ and its new label (after dropping $k$) be $l$. Then, if $l=k$ a Nash Equilibrium is reached. If $l\neq k$, then the algorithm continues by dropping another label and continuing until a completely labelled node has been reached, which is an equilibrium point. The starting label to drop is the pivot parameter as in the Matlab implementation. Since there are $m+e$ labels (strategies) which can be dropped first and they can end up in a different Nash Equilibrium node in the graph, it is inefficient to experiment with all different starting strategies. It will be assumed, that the default value 1 of the pivot is kept.

