\begin{otherlanguage}{english}
\begin{abstract}

This thesis proposes a new approach for solving the power distribution problem in a hybrid vehicle. In a typical parallel hybrid, torque is split between the internal combustion engine and the electric motor, which can work together in parallel. To solve the power management problem game-theoretical solution approaches are examined and a cooperative game between two players - the engine and the motor, is defined. Both players are allowed to make agreements between themselves by cooperating with each other. The strategies of the players in the game represent how much torque each player contributes at each time step. The utility functions of the players are minimized and are therefore regarded as penalties. They take into account torque deviation from the required torque, fuel consumption, CO, HC and NOX emissions and the State Of Charge (SOC) of the battery. These represent the goals of the game - meeting the demanded vehicle speed, while minimizing both fuel consumption, gas emissions and maintaining a target SOC of the battery. Six game-theoretical approaches applicable to cooperative game theory are explained, implemented and simulated - Pareto Optimality, Nash Equilibrium, Nash Bargaining solution, Kalai-Smorodinsky solution, Core and Shapley value. A Simulink model of a hybrid vehicle is simulated during the Federal Test Procedure 75 (FTP75) drive cycle with all six game theory approaches by solving the game at every stage. Results from all simulations are shown, compared and discussed.
  
\end{abstract}
\end{otherlanguage}
