%%%%%%%%%%%%%%%%%%%%%%%%%%%%%%%%%%%%%%%%%%%%%%%%%
% Configuration for Reports and bachelor thesis %
%                                               %
% designed by Tobias Schmidt                    %
% filename: ./config.tex                        %
%%%%%%%%%%%%%%%%%%%%%%%%%%%%%%%%%%%%%%%%%%%%%%%%%
%% BOOL

\usepackage{ifthen}                % Verwenden von bool'schen Variablen
\newboolean{draft}
\setboolean{draft}{false}          % F�r Finale Version: draft=false !!!
\newcommand{\draftfoot}{\ifthenelse{\boolean{draft}}{\fancyfoot[CO]{Entwurf vom \today}\fancyfoot[CE]{}}{\fancyfoot{}}}
%% END BOOL
\listfiles 
%% Schrift
\usepackage{gensymb}
\usepackage[latin1]{inputenc}       % Deklaration der Zeichenkodierung (u.a. f�r Umlaute)
\usepackage[T1]{fontenc}
\usepackage{float}
\usepackage[english,UKenglish]{babel}
%\renewcommand\dateUKenglish{\def\today{\number\day~%
% \ifcase \month \or January\or February\or March\or April\or May\or June\or
%   July\or August\or September\or October\or November\or December\fi\space
% \number\year}}

%\usepackage{dsfont}               % F�r eine mathematische Schrift
\usepackage{amssymb}              % F�r mathematische Symbole
\usepackage[tbtags]{amsmath}      % F�r mathematische Befehle
\usepackage{amsthm}
\let\mbf\mathbf
\renewcommand*{\mathbf}[1]{\boldsymbol{\mbf{#1}}}
%\newcommand{\thmautorefname}{Satz}
%\newtheorem{thm}{\thmautorefname}
%\usepackage{times}
\usepackage{helvet}               % Arial bei sffamily
\usepackage{xspace}               % Setzt Leerzeichen, wenn welche hingeh�ren
%\usepackage{lscape}             % Teile des Dokuments in Querformat
%\usepackage{longtable}            % Tabellen mit Seitenumbr�chen
\usepackage{multirow}             % Tabellen
%\usepackage{booktabs}             % -''-
%\usepackage{multicol}             % -''-
\usepackage{tabularx}             % -''-
\usepackage{array}
\newcolumntype{C}[1]{>{\centering\arraybackslash}p{#1}}
\usepackage[inline]{enumitem}     % Erm�glicht inline-lists  \begin{enumerate*}[label=\itshape\alph*\upshape)] \item ... \end{enumerate*}

\usepackage[
	german,                        % deutsche �berschrift
	intoc,                           % Im Inhaltsverzeichnis
%	noprefix,                      % ohne Sortiergruppenangabe
%	refpage,                       % mit Seitenangabe
]{nomencl}              % Abk�rzungsverzeichnis
\setlength{\nomitemsep}{-\parsep} % kein zus�tzlicher Abstand zwischen zwei Eintr�gen
\setlength{\nomlabelwidth}{6em}
\renewcommand{\nomname}{Abk�rzungs- und Zeichenverzeichnis}
\RequirePackage{ifthen}
\renewcommand{\nomgroup}[1]{%
\ifthenelse{\equal{#1}{A}}{\item[\hspace*{10pt}\textbf{Abk�rzungen}]\medskip}{%
\ifthenelse{\equal{#1}{B}}{\item[\hspace*{10pt}\textbf{Symbole}]\medskip}{%
\ifthenelse{\equal{#1}{C}}{\item[\hspace*{10pt}\textbf{Variablen}]\medskip}{}%
}}}
\makenomenclature
\newcommand{\abk}[2]{#1\nomenclature[A]{#1}{#2}}
\newcommand{\rabk}[2]{#1\nomenclature[A]{#2}{#1}}
\newcommand{\var}[2]{\nomenclature[C]{#1}{#2}}
\newcommand{\sym}[2]{\nomenclature[B]{#1}{#2}}

% Glossar mit referenzierter Seitenangabe (wirft Warnings bei mehreren Abk�rzungen auf der selben Seite)
%\renewcommand{\pagedeclaration}[1]{\hfill Seite \pageref{glossar#1}}
%\newcommand{\abk}[2]{#1\nomenclature{#1}{#2}\label{glossar\thepage}}
%\newcommand{\rabk}[2]{#1\nomenclature{#2}{#1}\label{glossar\thepage}}

%% Bibliographie mit BibLaTex und Biber
\usepackage[autostyle, german=quotes]{csquotes}

\usepackage[
backend=biber,
style=alphabetic,
style=authoryear,
natbib=true,
]{biblatex}
 
%\DefineBibliographyStrings{german}{andothers={et\addabbrvspace al\adddot}}

\usepackage[
	format=plain                % Text unter Bezeichner
	, labelfont=bf              % Bezeichner fett
	, font={
		it                      % Alles ist kursiv
		, small                 % Text kleiner
		, singlespacing         % Einfacher Zeilenabstand
	}
	, width=.9\textwidth,       % Schmaler als Textbreite
	, justification=raggedright % Text ist linksb�ndig
	, skip=5mm                  % Abstand von Beschriftung zu Objekt
]{caption,subcaption}           % Bild- und Tabellenunterschrift
\usepackage{flafter} % Place floatings after reference
%\captionsetup[sub]{
%    subrefformat=parens       % subref{} in Klammern
%}

%\usepackage{subfig}
%\newcommand{\subfigureautorefname}{\figureautorefname}

%% Grafiken und Farbe
\usepackage{graphicx}       % Einbindung von extern erzeugten .png Grafiken
\usepackage{xcolor}         % Farbige Schrift

\usepackage{fancyhdr}       % Kopf- und Fu�zeilen
\pagestyle{fancy}           % Unterkapitelzahl in Kopfzeile links, Kapitel rechts
\renewcommand{\chaptermark}[1]{\markboth{\textsc{\chaptertitlename} \thechapter{} -- #1}{}}
\renewcommand{\sectionmark}[1]{\markright{\thesection{} #1}{}}
\fancyhead[LO]{\leftmark}
\fancyhead[RO,LE]{\thepage}
\fancyhead[RE]{\rightmark}
\draftfoot
\fancypagestyle{plain}{%
	\fancyhf{}
	\renewcommand{\headrulewidth}{0pt}
	\renewcommand{\footrulewidth}{0pt}
\draftfoot
}
\fancypagestyle{empty}{%
\fancyhf{}
\renewcommand{\headrulewidth}{0pt}
\renewcommand{\footrulewidth}{0pt}
\draftfoot
}

\setcounter{secnumdepth}{2}           % Grad, bis zu dem Bezifferung erfolgt (0=Chapter, 1=Section, 2=Subsection, ...)
\setcounter{tocdepth}{2}              % Tiefe, bis zu der Eintrag ins Inhaltsverzeichnis
\usepackage[onehalfspacing]{setspace} % 1,5facher Zeilenabstand
\setlength{\parindent}{0pt}           % Erste Zeile im Absatz nicht einr�cken
\parskip 1.6ex                        % Abstand zwischen den Abs�tzen

%% Enable continuous footnote counting
\usepackage{remreset}
\makeatletter\@removefromreset{footnote}{chapter}\makeatother

% Disable single lines at the start of a paragraph (Schusterjungen)
\clubpenalty = 10000
% Disable single lines at the end of a paragraph (Hurenkinder)
\widowpenalty = 10000
\displaywidowpenalty = 10000

% Kontrolliert Silbentrennung ([none]: keine Trennung im gesamten Text, \nohyphens{lalala}: keine Silbentrennung in lalala)
% \usepackage{hyphenat}

% Weniger Worttrennungen, daf�r gr��erer Wortabstand
\sloppy

\renewcommand{\floatpagefraction}{.7} % Bild erst auf eigene Seite, wenn 60% der Seite einnehmen.

%% Quellcode
\usepackage{color, listings}          % Quellcode einbinden
\definecolor{olivegreen}{cmyk}{.25,0,1,.5}
\definecolor{froggreen}{cmyk}{1,0,1,.25}
\definecolor{dark_orange}{cmyk}{0,.5,1,.16}
\usepackage{courier}                  % Courier als tt-Schriftart
% http://blog.robert-kummer.de/2006/04/latex-quellcode-listing.html
\lstdefinestyle{none}{
	basicstyle=\ttfamily\scriptsize % Schriftart und -gr��e
	, breaklines=true               % Zeilen werden umgebrochen
%	, aboveskip=10pt                % Abstand nach oben
%	, belowskip=-1pt                % Abstand nach unten
	, fontadjust                    % Keine Blockausrichtung
	, showstringspaces=false        % Leerzeichen im String
	, captionpos=t                  % Beschreibung oben
	, tabsize=2
	, escapeinside={(*@}{@*)}       % Listing unterbrochen zwischen (*@ und @*)
}
\lstdefinestyle{default}{
	style = none
	, frame=single                  % Einfacher Rahmen
	, rulecolor=\color{black}       % Rahmenfarbe
	, numbers=left                  % Zeilennummern links
	, numberstyle=\tiny             % Zeilennummernschrift klein
	, numbersep=5pt                 % Zeilennummernabstand zum Listing
	, stringstyle=\color{orange}    % Farbe der Zeichenketten
	, commentstyle=\color{gray}     % Farbe der Kommentare
  , float=tbh                     % Packe Listing in Float-Umgebung
}
\lstdefinestyle{c}{
	style=default                       % Verwende defaultstyle
	, emphstyle=\bfseries\ttfamily
	, language=c                        % C
	, commentstyle=\color{froggreen}
	, keywordstyle=[1]\color{blue}
    , morekeywords=[1]{__attribute__, byte, GPS_Pos_t, WPoint_t,  UART_VersionInfo_t, GPS_PosDev_t, NaviData_t, WPointResponse_t, size_t}
    , emph={public, private, class}
}
\lstdefinestyle{line}{basicstyle=\ttfamily\small,breaklines=true,fontadjust}
\newcommand{\inlcode}[1]{\lstinline[style=line]|#1|}

% Paket f�r benutzerdefinierte Formatierung der �berschriften
\usepackage{titlesec}
\newcommand{\formatchaptertitle}[1]{\parbox[t]{\dimexpr\textwidth-10pt}{\raggedright#1}}
\titleformat{\chapter}[block]
{\Huge\bf\sffamily}
{\llap{\color{gray}\fontsize{60}{60}\usefont{U}{eur}{b}{n}\thechapter\hspace{10pt}\vline}}
{10pt}
{\formatchaptertitle}
\titleformat{\section}{\normalfont\Large\bfseries}{\thesection}{1em}{}
\titleformat{\subsection}{\normalfont\large\bfseries}{\thesubsection}{1em}{}
%% Links und Urls
\usepackage%
%[
%	pdftitle             = {Tobias\ Schmidt\ -\ Projektpraktikum}
%	, pdfauthor          = {Tobias Schmidt}
%	, breaklinks         = true
%]%
{hyperref}
\urlstyle{same}
\usepackage[
    open=true       % Lesezeichen aufklappen
    , openlevel=1   % Ge�ffnete Ebene der Lesezeichen (0=chapter, 1=section)
]{bookmark}

% Rename autorefnames
\addto\extrasngerman{%
\def\subsectionautorefname{\sectionautorefname}%
\def\subsubsectionautorefname{\sectionautorefname}%
}

\usepackage{scrhack}

\newcommand{\visit}[1]{; letzter Zugriff: #1}
\newcommand{\Kommentar}[1]{\ifthenelse{\boolean{draft}}{\emph{\color{gray} Anm.: #1}}{}}
\newcommand{\TODO}[1]{\ifthenelse{\boolean{draft}}{\par\emph{
\colorbox{pink}{\textbf{TODO #1}}}\par}{}}
\newcommand{\itembf}[1]{\item \textbf{#1}\newline}
\newcommand{\ignore}[1]{}

% Abk�rzungen
\newcommand{\bspw}{beispielsweise\xspace}
\newcommand{\ua}{u.\,a.\xspace}
%\newcommand{\idR}{i.\,d.\,R.\xspace}
\newcommand{\idR}{in der Regel\xspace}
\newcommand{\sog}{so genannte\xspace}
\newcommand{\zB}{z.\,B.\xspace}
\newcommand{\usw}{usw.\xspace}
\newcommand{\bzw}{bzw.\xspace}
\newcommand{\etal}{et al.\xspace}
\newcommand{\bzgl}{bzgl.\xspace}
