\begin{otherlanguage}{english}
\begin{abstract}

This thesis proposes a new approach for solving the power distribution problem in a hybrid vehicle. In a typical parallel hybrid, torque is split between the internal combustion engine and the electric motor, which can work together in parallel. The examined solution approaches are game-theoretical and the power management problem is defined as a cooperative game with two players - engine and motor, where both are allowed to make agreements by cooperation. The utility functions of the players take into account torque deviation from the required torque, fuel consumption, CO, HC and NOX emissions and the State of Charge (SOC) of the battery. These represent the goals of the game - meeting the demanded vehicle speed, while minimizing both fuel consumption, gas emissions and maintaining a target SOC of the battery. Six game-theoretical approaches applicable to cooperative game theory are explained, implemented and simulated - Pareto Optimality, Nash Equilibrium, Nash Bargaining solution, Kalai-Smorodinsky solution, Core and Shapley value. A Simulink model of a hybrid vehicle is simulated during the Federal Test Procedure (FTP75) drive cycle with all six game theory approaches by solving the game at every stage.
  
\end{abstract}
\end{otherlanguage}
